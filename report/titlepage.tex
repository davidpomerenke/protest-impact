\begin{textblock*}{\paperwidth}(0mm,0mm)
\begin{centering}
\begin{large}

\vspace{4cm}

Master Thesis

\vspace{1cm}

\rule{5cm}{0.4pt}

\vspace{0.5cm}

\textbf{
{\Huge How do protests shape discourse?}\\
\vspace{0.2cm}
{\huge
Causal methods for determining the impact\\
of protest events on newspaper coverage}
}

\vspace{0.5cm}

{\Large David Pomerenke}

\vspace{0.5cm}

\rule{5cm}{0.4pt}

\vspace{3cm}

Thesis submitted in partial fulfillment\\
of the requirements for the degree of\\
Master of Science of Artificial Intelligence\\
at the Department of Advanced Computing Sciences\\
of Maastricht University

\vspace{1cm}

\textbf{Thesis Committee:}\\
Christof Seiler, PhD\\
Marijn ten Thij, PhD

\vspace{3cm}

Maastricht University\\
Faculty of Science and Engineering\\
Department of Advanced Computing Sciences

\vspace{1cm}

August 30, 2023

\end{large}
\end{centering}
\end{textblock*}
\vspace*{\fill}
\thispagestyle{empty}
\newpage

\textbf{Abstract.} Protests can have an impact on newspaper coverage, not only by prompting reports about the protest events themselves, but also by bringing attention to the issue that they are concerned with. But they may also have negative impacts by distracting from existing constructive discourse on the issue. Quantitative media analyses can uncover these impacts. The problem is that protests and media coverage are in a complex causal relation: They mutually influence each other, and external events may cause both protests and coverage to increase at the same time. To deal with observed and unobserved confounding, I evaluate multiple causal methods: Besides the classical repertoire of regression and instrumental variables, I investigate aggregated synthetic controls and inverse propensity weighting. I show that all methods reduce bias but do not completely remove it, except perhaps the synthetic control method. My analysis of climate protest events in Germany shows that the protests generally tend to increase not only protest-related coverage but also other coverage related to climate change; but there are differences between the various protest groups. This sheds empirical light on a theoretical debate about possible backfiring effects.

\textbf{Acknowledgements.} Many thanks to my supervisors, Christof Seiler and Marijn ten Thij, for their support, their ideas, their encouragement, and their detailed feedback; to Anda Iamnitchi, Eli Sapir, Victor Wang, and Ben Laurense for their technical suggestions; to James Ozden, Markus Ostarek, and Lennart Klein for inspiration and fruitful exchanges; and to Amir, Iris, and my parents for their curiosity, encouragement, and support.

\newpage
